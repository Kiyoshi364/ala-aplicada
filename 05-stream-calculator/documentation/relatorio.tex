\documentclass{article}

\usepackage[a4paper]{geometry}

\usepackage[brazil]{babel}
\usepackage[T1]{fontenc}

\usepackage{amsmath}
\usepackage{amssymb}

\usepackage{hyperref}

\usepackage{multicol}

% noindent EVERYWHERE
\setlength{\parindent}{0pt}

\newcommand{\To}{\Rightarrow}

\newcommand{\head}{\emph{head}}
\newcommand{\tail}{\emph{tail}}
\newcommand{\Head}{\emph{Head}}
\newcommand{\Tail}{\emph{Tail}}

\newcommand{\coisa}[1]{\textbf{coisa#1}}

\newcommand{\note}[1]{\textbf{Nota:} \textit{#1}}

\newcommand{\Num}{\mathcal{N}}

\newcommand{\ins}[1]{\langle #1 \rangle}
\newcommand{\inv}[1]{{#1}^{-1}}

\newcommand{\dre}{119025937}
\newcommand{\hashi}{Daniel Kiyoshi Hashimoto Vouzella de Andrade}

\title{Calculadora de Streams}
\author{\hashi{} -- \dre{}
}
\date{Julho 2023}

\begin{document}
\maketitle

\section{Por quê Streams?}

Streams são sequências infinitas de \coisa{s}
\[
    (a, \; b, \; c, \; d, \; e, \; f, \; \dots)
\]

Aparecem em Computação:
\begin{itemize}
    \item Stream de dados
    \begin{itemize}
        \item Vídeo
        \item Áudio
        \item \dots
    \end{itemize}

    \item Processamento de Sinais
    \begin{itemize}
        \item Filtros, conversores, \dots
    \end{itemize}
\end{itemize}

Aparecem em Matemática:
\begin{itemize}
    \item Séries de Potência/Polinômios
    \[
        a_0 \cdot x^0 + a_1 \cdot x^1 + a_2 \cdot x^2 + \cdots
    \]
    \item Série de Taylor
    \[
        f(x) =
        f(a)
        + \frac{f'(a) \cdot (x-a)^1}{1!}
        + \frac{f''(a) \cdot (x-a)^2}{2!}
        + \cdots
    \]
    \item Problemas de contagem (sequências aparecem como solução)
    \begin{itemize}
        \item Fibonacci
        \[
            0, \; 1, \; 1, \; 2, \; 3, \; 5, \; 8, \; \dots
        \]
        \item Números Triangulares
        \[
            1, \; 3, \; 6, \; 10, \; \dots, \; \frac{n \cdot (n+1)}{2}, \; \dots
        \]
    \end{itemize}
\end{itemize}

\section{O que são Streams?}

Matematicamente,
uma Stream de \(A\) é um elemento do conjunto:
\[
    A^\omega := \{
        S \;;\;
        S \in \{ 0, \; 1, \; 2, \; \dots \} \to A
    \}
\]
ou seja, funções de \(\mathbb{N}\)
para seus elementos de \(A\).

Mas podemos fingir que são uma `lista infinita' de \(A\)s:
\[
    S := (s_0, \; s_1, \; s_2, \; s_3, \; s_4, \; s_5, \; \dots)
\]

\subsection{O que podemos fazer com Streams?}

Temos algumas operações úteis:
\begin{itemize}
    \item Indexar:
        pedir o \(i\)-ésimo elemento da stream
    \begin{align*}
        S[i] &:= s_i
    \end{align*}

    \emph{Quando \(i = 0\) chamamos a operação de \head{}.}
    \item Tail/Derivar/Shift Left:
        jogar o primeiro elemento \coisa{} fora
    \begin{align*}
        S' &:= (s_1, \; s_2, \; s_3, \; s_4, \; s_5, \; \dots)
        \\ \\
        S'[i] &:= S[i+1]
    \end{align*}

    \item Cons/Integrar/Shift Right:
        colocar uma \coisa{} no início da Stream
    \begin{align*}
        (a : S) := (a, \; s_0, \; s_1, \; s_2, \; s_3, \; \dots)
        \\ \\
        \begin{cases}
            (a : S)[0] := a \\
            (a : S)[i] := S[i-1]
        \end{cases}
    \end{align*}
\end{itemize}

\subsection{Como definimos Streams?}

Podemos usar \emph{Equações Diferenciais}
para definir Streams.

Para isso, basta dizer quem é
\Head{} e \Tail{} de \(S\).

Alguns exemplos:
\begin{multicols}{2}
\begin{itemize}
    \item ``Sempre'' \(a\):
    \begin{align*} \begin{cases}
        S[0] &= a \\
        S'   &= S
    \end{cases} \end{align*}

    \item ``Sempre'' \(a\) e \(b\):
    \begin{align*} \begin{cases}
        S [0] &= a \\
        S'[0] &= b \\
        S''   &= S
    \end{cases} \end{align*}

    \item ``Sempre'' \(a\) e \(b\) (versão 2):
    \begin{align*} \begin{cases}
        S[0] &= a \\
        T[0] &= b \\
        S'   &= T \\
        T'   &= S
    \end{cases} \end{align*}

    \item Os pares de \(S\):
    \begin{align*} \begin{cases}
        par(S)[0] &= S[0] \\
        par(S)'   &= par(S'')
    \end{cases} \end{align*}

    \item Os ímpares de \(S\):
    \begin{align*} \begin{cases}
        impar(S)[0] &= S[1] \\
        impar(S)'   &= impar(S'')
    \end{cases} \end{align*}

    \item Zip de \(S\) e \(T\):
    \begin{align*} \begin{cases}
        zip(S, T)[0] &= S[0] \\
        zip(S, T)'   &= zip(T, S')
    \end{cases} \end{align*}

    \item Double de `S`:
    \begin{align*} \begin{cases}
        double(S)[0] &= S[0] \\
        double(S)[1] &= S[0] \\
        double(S)''  &= double(S')
    \end{cases} \end{align*}
\end{itemize}
\end{multicols}

Com essas definições,
podemos ir calculando \head{}s e \tail{}s
e começar a montar a lista infinita.

E após algumas tentativas as vezes conseguimos
adivinhar a ``cara'' dessas Streams.

Assim são o início dessas streams:
\begin{multicols}{2}
\begin{itemize}
    \item ``Sempre'' \(a\):
    \[
        (a, \; a, \; a, \; \dots)
    \]

    \item ``Sempre'' \(a\) e \(b\):
    \[
        (a, \; b, \; a, \; b, \; a, \; b, \; \dots)
    \]

    \item Pares de \(S\):
    \[
        (s_0, \; s_2, \; s_4, \; s_6, \; \dots)
    \]

    \item Impares de \(S\):
    \[
        (s_1, \; s_3, \; s_5, \; s_7, \; \dots)
    \]

    \item Zip de \(S\) e \(T\):
    \[
        (s_0, \; t_0, \; s_1, \; t_1, \; s_2, \; t_2, \; \dots)
    \]

    \item Double de \(S\):
    \[
        (s_0, \; s_0, \; s_1, \; s_1, \; s_2, \; s_2, \; \dots)
    \]
\end{itemize}
\end{multicols}

\subsection{Equivalência de Streams com Autômatos}

Um autômato é uma quadrupla \((Q, q_0, out, next)\),
onde:
\begin{itemize}
    \item \(Q\): Conjunto de Estados (possívelmente infinito)
    \item \(q_0\): O primeiro estado
    \item \(out\): Função \(Q \to A\)
    \item \(next\): Função \(Q \to Q\)
\end{itemize}

A ideia é que cada estado
gera um elemento (usando \(out\)) e
\(next\) diz o resto da Stream.

\(out\) está relacionada com \head{} e
\(next\) está relacionada com \tail{}.

Usando essa ideia,
podemos pegar uma Stream
\(S \in A^\omega\) e
fazer um autômato equivalente
\[
    (S, s_0, head, tail)
\]
onde:
\begin{align*}
    s_0     &:= S[0] \\
    head(T) &:= T[0] \\
    tail(T) &:= T'
\end{align*}

\section{Streams de Números: um caso especial}

Daqui para frente vamos usar \emph{Números Racionais}
\(\Num\) no lugar de \(A\).
Mas vai funcionar com qualquer outro \emph{Corpo}.

\subsection{Operações básicas}

\subsubsection{Insersão}

Podemos levar elementos de \(\Num\)
em elementos de \(\Num^\omega\):
\begin{align*} \begin{cases}
    \ins{n}[0] &:= n \\
    \ins{n}'   &:= \ins{0}
\end{cases} \end{align*}

Intuitivamente:
\begin{align*}
    \ins{n} := (n, \; 0, \; 0, \; 0, \; \dots)
\end{align*}

\subsubsection{Soma}

Podemos somar dois \(\Num^\omega\),
\(S\) e \(T\),
resultando em outro \(\Num^\omega\):
\begin{align*} \begin{cases}
    (S + T)[0] &:= S[0] + T[0] \\
    (S + T)'   &:= (S' + T')
\end{cases} \end{align*}

Intuitivamente:
\begin{align*}
    S + T := (s_0 + t_0, \; s_1 + t_1, \; s_2 + t_2, \; \dots)
\end{align*}

Observe que se ``somarmos'' com um \(n \in \Num\):
\begin{align*}
    \ins{n} + S := (n + s_0, \; s_1, \; s_2, \; \dots)
\end{align*}

\subsubsection{Produto}

Podemos multiplicar dois \(\Num^\omega\),
\(S\) e \(T\),
resultando em outro \(\Num^\omega\):
\begin{align*} \begin{cases}
    (S \times T)[0] &:= S[0] \cdot T[0] \\
    (S \times T)'   &:= (\ins{S[0]} \times T') + (S' \times T)
\end{cases} \end{align*}

Esse é menos intuitivo,
mas fica parecido com
a propriedade distributiva da multiplicação de números:
\begin{align*}
    S \times T :=
    (s_0 \cdot t_0,
    \; s_1 \cdot t_0 + s_0 \cdot t_1,
    \; s_0 \cdot t_2 + s_1 \cdot t_1 + s_2 \cdot t_0,
    \; \dots)
\end{align*}

Observe que se ``multiplicarmos'' com um \(n \in \Num\):
\begin{align*}
    \ins{n} \times S := (n \cdot s_0, \; n \cdot s_1, \; n \cdot s_2, \; \dots)
\end{align*}

\subsection{Semelhança com Polinômios (Stream X)}

No geral, a Stream
\[
    S = (s_0, \; s_1, \; s_2, \; \dots)
\]
é muito parecida com o polinômio
\[
    P(x) = s_0 \; x^0 + s_1 \; x^1 + s_2 \; x^2 + \cdots
\]

Temos uma Stream especial,
vamos chamar de \(X \in \Num^\omega\),
que façam Streams mais parecidas com polinômios?
Queremos algo assim:
\begin{align*}
    \ins{s_0} &= (s_0, \; 0, \; 0, \; \dots) \\
    \ins{s_1} \times X &= (0, \; s_1, \; 0, \; \dots) \\
    \ins{s_2} \times X \times X &= (0, \; 0, \; s_2, \; \dots) \\
    \vdots&
\end{align*}

Temos!
\[
    X := (0, \; 1, \; 0, \; 0, \; 0, \; \dots)
\]

Observe que multiplicar por \(X\)
é equivalente a ``empurar'' a Stream
toda para direita
e por um 0 como primeiro elemento:
\[
    S \times X
    = (0, \; s_0, \; s_1, \; s_2, \; s_3, \; \dots)
    = (0 : S)
\]

\subsection{Recorrência Linear}

Podemos usar Streams para resolver Recorrências Lineares.

O exemplo clássico é a recorrência de Fibonacci:
\begin{align*}
    F[0] &:= 0 \\
    F[1] &:= 1 \\
    F[n] &:= F[n-1] + F[n-2]
\end{align*}

Podemos fazer uma versão ``recursiva'':
\begin{align*}
    F &:=
        (0 + \ins{1} \times X)
        + (F \times X)
        + (F \times X \times X) \\
    &:=
        (\ins{1} \times X)
        + (F \times X)
        + (F \times X \times X) \\
    &:=
        X
        + (F \times X)
        + (F \times X \times X) \\
\end{align*}

No desenho ficaria algo como:
\[
    \begin{array}{ccccccc}
     & (  0, &   1, &   0, &   0, &   0, & \dots) \\
    +& (  0, & f_0, & f_1, & f_2, & f_3, & \dots) \\
    +& (  0, &   0, & f_0, & f_1, & f_2, & \dots) \\ \hline
    =& (f_0, & f_1, & f_2, & f_3, & f_4, & \dots) \\
    \end{array}
\]

No olho, podemos descobrir que:
\begin{align*}
    f_0 &:= 0 \\
    f_1 &:= 1 + f_0 \\
    f_2 &:= f_0 + f_1 \\
    &\quad\vdots
\end{align*}

Mas também podemos resolver algebricamente:
\begin{align*}
    &F = X + F \times X + F \times X \times X \\
    &F = X + F \times (X + X \times X) \\
    &F - F \times (X + X \times X) = X \\
    &F \times (1 - X - X \times X) = X \\
\end{align*}

Agora falta ``jogar aquilo para o outro lado''.
\[
    F = \frac{X}{1 - X - X \times X}
\]

E pronto!

Mas como funciona essa \emph{matemágica}?

\subsection{Inversa de uma Stream}

Queremos uma operação de ``inversa''
com a seguinte propriedade:
\[
    S \times \inv{S} = \ins{1} = \inv{S} \times S
\]

A seguinte definição funciona:
\begin{align*} \begin{cases}
    (\inv{S})[0] &:= \frac{1}{S[0]} \\
    (\inv{S})'   &:= \ins{\frac{1}{S[0]}} \times S' \times \inv{S}
\end{cases} \end{align*}

Observe que \(\inv{S}\) não está definida
quando \(S[0] = 0\).

\iffalse
\section{Streams Racionais}

Com isso completamos o que precisamos para
definir \textbf{Streams Racionais}.

Uma \textbf{Stream Racional} \(A^\omega\) é uma generalização
da \textbf{Streams de Números} da seção anterior,
podemos:
\begin{itemize}
    \item Somar
    \item Multiplicar
    \item Inverter
\end{itemize}
\note{Genericamente só pedimos que \(A\) forme um Corpo.}

\subsection{Circuitos}
\fi

\section{A calculadora}

A calculadora é um programa iterativo.
Ela recebe comandos na notação polonesa reversa,
em outras palavras,
recebe os argumentos e depois a operação.

Por exemplo:
\[
    13 \; 2 + 4 \; 6 * \; -
\]
é equivalente à
\[
    (13 + 2) - (4 * 6)
\]

Como ela só trabalha com streams,
números são automaticamente inseridos em streams.

O código foi desenvolvido em \texttt{C},
e única dependência é um compilador de \texttt{C}.
Foi usado o \texttt{gcc} na versão \texttt{11.3.0}.

Existem alguns exemplos em \texttt{examples.txt}.
O arquivo é uma entrada válida para a calculadora.

\subsection{Como rodar}

Existe um script para compilar e rodar o programa:
\texttt{compile\_run.sh}.
O programa é pequeno o suficiente para
compilar o programa logo antes de executar
não ser um problema.

Para a compilação ele pode receber dois valores
que configuram a saída da calculadora:
\begin{itemize}
    \item \texttt{-DPRINT\_COUNT=<N>}: padrão 5. \par
    Diz quantos elementos da Stream serão mostrados
    quando ela mostra as Streams que estão na pilha.
    \item \texttt{-DPRINT\_TAIL\_DEPTH=<N>}: padrão 0. \par
    Diz o quão profundo o programa tenta ao mostrar a calda da stream
    quando ela mostra as Streams que estão na pilha.
    \item \texttt{-DDEBUG}\par
    Liga alguns prints extras para ajudar a debugar a calculadora.
\end{itemize}

Exemplo, usando as flags:
\begin{verbatim}
$ ./compile_run.sh -DPRINT_COUNT=10 -DPRINT_TAIL_DEPTH=2
\end{verbatim}

Geralmente não é necessário se preocupar com isso.

O programa pode ser encerrado com um EOF
(em linux \texttt{Ctrl-D}).
Não tem um comando \texttt{exit} ou similar que encerra a calculadora.

\subsubsection{Testes}

Existe uma pequena bateria de testes.
Para rodar basta executar:
\texttt{run-tests.sh}.

Os arquivos do teste estão na pasta \texttt{tests/}.

\subsection{Sintaxe e Comandos}

A sintaxe é permissiva,
isso quer dizer que:
na dúvida,
os ``erros'' serão ignorados
ou silenciosamente convertidos para coisas válidas.

Isso não é uma coisa ideal,
mas contorna alguns problemas e
permite ``focar no que importa''
(fazer a calculadora funcionar).

Além disso, ela é case-insensitive,
ou seja, os seguintes nomes são equivalentes:
\begin{multicols}{4}
\begin{itemize}
    \item \texttt{add}
    \item \texttt{Add}
    \item \texttt{aDd}
    \item \texttt{ADD}
\end{itemize}
\end{multicols}

\subsubsection{Números}

Ao ler um número, converte ele para stream e o empilha.

Números começam com um underscore (\texttt{\_}) ou digito,
aceita vários deles e opcionalmente um ponto (\texttt{.}) e
mais underscore (\texttt{\_}) ou digito.

Se o número começa com underscore (\texttt{\_})
é um número negativo.

Exemplos:
\begin{multicols}{2}
\begin{itemize}
    \item \texttt{0}: \(0\)
    \item \texttt{5}: \(5\)
    \item \texttt{\_2}: \(-2\)
    \item \texttt{10\_000\_123.456\_7}: \(10000123.4567\)
    \item \texttt{\_32.5}: \(-32.5\)
    \item \texttt{\_}: \(0\)
\end{itemize}
\end{multicols}

\subsubsection{Operações de Streams}

As operações retiram streams da pilha
e empilha o resultado na pilha.

Se não tem argumentos o suficiente na pilha,
não fazem nada.

Geralmente, o comando pode ser executado
usando um símbolo ou um nome.

\paragraph{Soma}

Pode ser invocada com \texttt{+} ou \texttt{add}.

\paragraph{Subtração}

Pode ser invocada com \texttt{-} ou \texttt{sub}.

\paragraph{Multiplicação}

Pode ser invocada com \texttt{*} ou \texttt{mul}.

\paragraph{Inversa}

Pode ser invocada com \texttt{\%} ou \texttt{inv}.

\paragraph{Derivada/Shift Left}

Pode ser invocada com \texttt{\textquotesingle{}} ou \texttt{tail}.

\paragraph{Integral/Shift Right}

Essa operação é equivalente a multiplicar
por \(X^n\), ou seja, \(X\) multiplicado por ele mesmo \(n\) vezes.
Pode ser invocada com \texttt{\textasciicircum{}n},
onde \texttt{n} é um inteiro (preferencialmente maior que 0).

\paragraph{Stream X}

Esse é um atalho para empilhar a stream \(X\).
Pode ser invocada com \texttt{x}.

\note{Prefira usar \texttt{\textasciicircum{}n}
no lugar de várias multiplicações por \texttt{x}.
O \emph{Shift Right} tem algumas otimizações
que economizam memória e tempo de execução.
Mas para exemplos pequenos, não faz muita diferença.}

\subsubsection{Utilitários da calculadora}
\paragraph{Escreve em Registrador}

A calculadora possui 10 registradores,
numerados de 0 até 9.
Pode escrever em um deles usando \texttt{!r},
onde \texttt{r} é um inteiro (preferencialmente no intervalo \([0 , 9]\)).

Escrever em um registrador pode ser usado
para jogar o topo da pilha fora.

\paragraph{Le do Registrador}

Pode empilhar o conteúdo de um registrador usando \texttt{\$r},
onde \texttt{r} é um inteiro (preferencialmente no intervalo \([0 , 9]\)).

Os registradores começam inicializados com a Stream \(\ins{0}\).

\paragraph{Copia da Pilha}

Pode copiar um valor da pilha para o topo dela usando \texttt{@s},
onde \texttt{s} é um inteiro (dentro do tamanho da pilha,
aceita números negativos).

\subsubsection{Outros}
\paragraph{Comentários}

Um comentário começa com \texttt{\#} e vai até o fim da linha.

\paragraph{Não mostra pilha}

Se a linha termina com \texttt{;},
não mostra a pilha.

Isso pode ser útil para não ficar reavaliando
os valores das Streams empilhadas.

\paragraph{Mostra registradores}

Se a linha termina com \texttt{;},
não mostra a pilha,
mas imprime os registradores que não possuem
a Stream \(\ins{0}\).

\subsubsection{Ignorados}

Outros símbolos e palavras
não mapeados são ignorados.

Exemplos:
\begin{multicols}{3}
\begin{itemize}
    \item \texttt{[}
    \item \texttt{]}
    \item \texttt{\{}
    \item \texttt{\}}
    \item \texttt{\textasciitilde}
    \item \texttt{lalala}
    \item \texttt{arroz}
    \item \texttt{addi}
    \item \texttt{c}
\end{itemize}
\end{multicols}

\subsection{Alguns detalhes de implementação}

\subsubsection{Limites arbitrários que podem ser alterados}
\begin{itemize}
    \item Limite de registradores: 10
    \item Limite de tamanho da pilha: 16 (0x10)
    \item Limite de tamanho de uma linha da entrada: 1024 (0x400)
\end{itemize}

\subsubsection{Arquitetura}

A calculadora possui 3 arquivos principais:
\texttt{streams.h}, \texttt{lexer.h} e \texttt{main.c}.

\paragraph{\texttt{streams.h}}

Aqui está toda a implementação de Streams de floats.
Streams são representadas como ponteiros para blocos imutáveis
de memória.

Elas tem uma ``tipagem dinâmica'',
um campo que diz ``qual tipo ela é''.
Isso permite tratar tudo como se ``fosse a mesma coisa''
e algumas otimizações,
mas a cada operação tem que \emph{ver} qual
``tipo'' cada coisa é e isso é meio chato
(nome disso é \emph{dynamic dispatch}).

\paragraph{\texttt{lexer.h}}

Essa parte ``tokeniza'' a entrada:
lê e segmenta a entrada
para um formato mais fácil do
programa compreender (\emph{tokens}).

\paragraph{\texttt{main.c}}

É a ``cola'' de tudo:
recebe a entrada,
passa para o Lexer,
interpreta os \emph{tokens},
realiza as operações de Streams.

\subsubsection{Possíveis Melhorias}
\paragraph{Reuso de memória}

Esse programa só aloca memória indefinidamente,
nunca se preoculpa em ``devolvê-la'' ou reutilizá-la.
Em teoria, apenas uma stream \emph{não-trivial}
sendo exibida várias vezes na tela,
pode causar um estouro de memória.

Mas na prática isso não é tão problemático assim,
uma Stream ocupa apenas \textbf{32 bytes}
mais uma memória compartilhada para cada
Stream que ela depende
(uma soma depende de duas outras streams,
por exemplo).
Dessa forma, \((\ins{0} + (\ins{0} * \ins{0}))\)
ocuparia \(\mathbf{32 \cdot 3 = 96} \textbf{ bytes}\)
se compartilhasse todos os \(\ins{0}\)
e não simplificasse a conta para \(\ins{0}\).

\paragraph{Double dispatch}

As vezes é possível simplificar as streams
usando propriedades conhecidas, como:
\begin{itemize}
    \item \(\ins{a} + \ins{b} \To \ins{a + b}\)
    \item \(\ins{a} \times \ins{b} \To \ins{a \cdot b}\)
    \item \((A \times B) + (A \times C) \To A \times (B + C)\)
    \item \((A \times X) \times X \To A \times X^2\)
        \par\note{\(A \times X^2\) tem uma representação melhor
        do que \(A \times X \times X\),
        por isso existe operação \texttt{Shift Right}.}
    \item \((A \times X) + (B \times X^2)
        \To (A + (B \times X)) \times X\)
    \item \(\cdots\)
\end{itemize}

\paragraph{Mais ``Tipos'' de Streams}

Existem outras representações específicas de Streams
que podem ser úteis.
Algumas ideias:
\begin{itemize}
    \item \texttt{Loop(L, i)}:
        guarda uma lista de elementos \texttt{L} e
        um índice \texttt{i}: \par
        \Head{} é \texttt{L[i]} e
        \Tail{} é \texttt{Loop(L, (i+1) \% len(L))}.
    \item \texttt{Autom(L, i, r)}:
        guarda uma lista de elementos \texttt{L},
        dois índices \texttt{i} e \texttt{r}: \par
        \Head{} é \texttt{L[i]} e
        \Tail{} é \texttt{if i+1 < len(L)
            then
                Autom(L, i+1, r)
            else
                Autom(L, r, r)}.

        Deve ser valer a equivalencia:
        \texttt{Loop(L, i) = Autom(L, i, 0)}.

        \note{Não foi provado,
        mas acredito que os Autômatos
        sempre podem ser transformados
        em uma sequência de elementos
        com uma indicação para qual elemento
        ela deve recomeçar logo depois de chegar ao fim
        (um desenho agora seria muito bom).}
    \item \textbf{Alguma representação mágica de Razão de Polinômios}:
        \par Como \textbf{Streams Racionais} podem ser
        representadas como Razões de Polinômios,
        talvez tenha alguma forma interessante de
        se calcular \Head{} e \Tail{} a partir disso.

        Talvez isso tenha alguma representação interessante
        usando duas listas
        (uma para a parte ``de cima'' e
        outra para a parte ``de baixo'' da razão).
\end{itemize}

\paragraph{Lexer}

Algumas entradas não esperadas quebram o Lexer.
Geralmente elas são não-letras e não-digitos,
acentos/letras com acentos e caracteres de outras línguas.

\newpage
\nocite{*}
\bibliographystyle{plain}
\bibliography{relatorio}

\end{document}
