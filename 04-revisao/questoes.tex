\documentclass{article}

\usepackage[a4paper]{geometry}

\usepackage[brazil]{babel}
\usepackage[T1]{fontenc}

\usepackage{amsmath}
\usepackage{amssymb}

\usepackage{multicol}

\usepackage{ifthen}

\newcommand{\R}{\mathbb{R}}

\DeclareMathOperator*{\cond}{cond}

% noindent EVERYWHERE
\setlength{\parindent}{0pt}

\newenvironment{question}
    {\medskip\bfseries\large}
    {\medskip}

\newcounter{exe-list}
\newenvironment{exe-list}
    {\begin{list}{(\alph{exe-list})}{\usecounter{exe-list}}}
    {\end{list}}

\newenvironment{exe}[2][Sala]
    {\bigskip\noindent\par\ifthenelse{\equal{#1}{}}%
        {\textbf{\LARGE #2}}%
        {\textbf{\LARGE #1~#2}}%
    \medskip\noindent\par}
    {\bigskip}

\newcommand{\todo}[1]{\textbf{TODO:} {\itshape #1}}

\title{Lista de Revisão}
\author{Nome -- DRE}
\date{Álgebra Linear Aplicada - 2023.1}

\begin{document}
\maketitle

\begin{exe}{1}
    \begin{question}
        Calcule:
        \[
            \min_x \left\|
                a - x \; b
            \right\|^2
        \]
        onde:
        \[
            a = \begin{bmatrix}
                a_0 \\ a_1 \\ a_2
            \end{bmatrix}
            \qquad\text{,}\qquad
            b = \begin{bmatrix}
                1 \\ 1 \\ 1
            \end{bmatrix}
            \qquad\text{e}\qquad
            a_0, a_1, a_2, x \in \R
        \]
    \end{question}
\end{exe}

\begin{exe}{2}
    \begin{question}
        Calcule:
        \[
            \min_x \left\|
                a - x \; b
            \right\|^2
        \]
        onde:
        \[
            a, b \in \R^n
            \qquad\text{e}\qquad
            x \in \R
        \]
    \end{question}
\end{exe}

\begin{exe}{3}
    \begin{question}
        (Regressão Rigde) Calcule:
        \[
            \min_x \left\|
                A \; x - b
            \right\|^2
            + \lambda \; \|x\|^2
        \]
        onde:
        \[
            A \in \R^{m \times n}
            \qquad\text{,}\qquad
            b \in \R^m
            \qquad\text{,}\qquad
            x \in \R^n
            \qquad\text{e}\qquad
            \lambda \in \R
        \]
    \end{question}
\end{exe}

\begin{exe}{4}
    \begin{question}
        Se matrix \(A \in \R^{N \times N}\)
        tem condicionamento \(10^3\).
        Qual é condicionamento da matriz \(A^T A\)?
        O novo condicionamento é pior ou melhor
        do que a de \(A\) para se usar no computador?

        \medskip

        Lembrete: \(\cond(A) = \frac{\sigma_1}{\sigma_N}\)
    \end{question}
\end{exe}

\begin{exe}{5}
    \begin{question}
        Qual seria a saída do algorítmo K-means
        na notação matricial para esse conjunto de pontos
        com \(K = 2\)?
        \begin{multicols}{2} \begin{itemize}
            \item[] \(a_1 := (0, 0)\)
            \item[] \(a_2 := (0, 1)\)
            \item[] \(a_3 := (8, 4)\)
            \item[] \(a_4 := (-3, 0)\)
            \item[] \(a_5 := (9, 3)\)
        \end{itemize} \end{multicols}
    \end{question}
\end{exe}

\begin{exe}{6}
    \begin{question}
        O float do seu computador tem erro de \(10^{-8}\)
        e você só pode cometer erros na segunda casa decimal.
        Qual é um condicionamento de \(A\)
        problemático para você?
    \end{question}
\end{exe}

\begin{exe}{7}
    \begin{question}
        Como os pontos \(a_1, \dots, a_{10}\)
        estão distribuídos em \(\R^2\),
        se o seu dendograma é:

        \todo{incluir desenho}
    \end{question}
\end{exe}

\begin{exe}{8}
    \begin{question}
        (Coeficiente de Rayleign) Calcule:
        \[
            \max_x \frac{ x^T A x }{ x^T x }
        \]
        onde:
        \[
            A \in \R^{n \times n}
            \qquad\text{,}\qquad
            A = A^T
            \qquad\text{e}\qquad
            x \in \R^n
        \]
    \end{question}
\end{exe}

\begin{exe}[Lista]{1}
    \begin{question}
        Num país politicamente instável,
        \(30\%\) dos defensores da república
        passam a apoiar a monarquia a cada ano e
        \(20\%\) dos defensores da monarquia
        passam a apoiar a república a cada ano.
        Portanto, denotando por \(r_l\) e \(m_k\)
        o número de republicanos e monarquistas,
        respectivamente, no ano k.
        \begin{exe-list}
            \item
                Qual é o código para calcular
                \(r_k\) e \(m_k\)?
            \item
                Sabendo que hoje metade da população
                apoia a república,
                em 10 anos
                qual será o percentual que apoia a república?
            \item
                A longo prazo qual será
                o percentual de republicanos e monarquistas?
        \end{exe-list}
    \end{question}
\end{exe}

\begin{exe}[Lista]{2}
    \begin{question}
        Sequência de Fibonacci. \medskip\noindent\par
        A sequência de Fibonacci
        é definida pelas fórmulas:
        \begin{align*}
            F_0 &= 0 \\
            F_1 &= 1 \\
            F_{t+1} &= F_t + F_{t-1}
        \end{align*}
        Os \(13\) primeiros números da sequência são
        \(0, 1, 1, 2, 3, 5, 6, 13, 21, 34, 55, 89, 144\).

        Esta famosa sequência tem uma profunda conexão
        com o número irracional \(\phi\),
        conhecido como Proporção Áurea.
        Esta proporção possui a seguinte propriedade geométrica:

        \todo{incluir desenho}
        \[
            \frac{a}{b} = \phi = \frac{a+b}{a}
        \]
        \begin{exe-list}
            \item
                Seja
                \[
                    v = \begin{bmatrix}
                        F_t \\ F_{t+1}
                    \end{bmatrix}
                \]
                um vetor cuja primeira coordenada é
                um elemento da sequência e
                a segunda coordenada é o elemento seguinte.
                Determine qual é a matriz \(A\) que
                avança o vetor \(v\) ao longo da sequência,
                ou seja,
                \[
                    A \; v
                    = A \; \begin{bmatrix}
                        F_t \\ F_{t+1}
                    \end{bmatrix}
                    = \begin{bmatrix}
                        F_{t+1} \\ F_{t+2}
                    \end{bmatrix}
                \]
            \item
                Determine os autovetores e autovalores
                da matriz \(A\).
                Sabendo que o resultado da aplicação repetida
                de uma transformação linear
                tende ao autovetor de maior autovalor associado
                daquela transformação
                (\emph{Método da Potência}),
                escreva em Português o que
                os autovetores e autovalores nos dizem
                sobre a Sequência de Fibonacci e
                sua relação com a Proporção Áurea.
            \item
                Dada a lista de números
                da Sequência de Fibonacci acima,
                confira se as conclusões às quais você chegou
                no item anterior se verificam.
        \end{exe-list}
    \end{question}
\end{exe}

\begin{exe}[Lista]{3}
    \begin{question}
        População de bactérias. \medskip\noindent\par
        A população de uma certa espécie de bactéria
        pode ser compreendida da seguinte maneira.
        Existem bactérias novas, maduras e velhas.
        A cada mês:
        \begin{list}{\(\bullet\) (\arabic{exe-list})}{\usecounter{exe-list}}
            \addtocounter{exe-list}{-1}
            \item
                \(80\%\) das bactérias novas
                chegam à maturidade, e
                \(20\%\) morrem;
            \item
                \(50\%\) das bactérias maduras
                tornam-se velhas, e
                \(50\%\) morrem;
            \item
                \(100\%\) das bactérias velhas morrem;
            \item
                Uma a cada duas bactérias maduras
                geram uma nova bactéria;
            \item
                Uma a cada cinco bactérias velhas
                geram uma nova bactéria.
        \end{list}

        \begin{exe-list}
            \item
                Modele o sistema populacional descrito acima --
                ou seja, determine o significado de
                cada coordenada do vetor que representa a população
                em um dado mês,
                e a matriz que representa
                a transição de um mês para o seguinte.
        \end{exe-list}
    \end{question}
\end{exe}

\begin{exe}[Lista]{8}
    \begin{question}
        Sejam
        \[
            x = \begin{bmatrix}
                0 \\ 1
            \end{bmatrix}
            \qquad\text{e}\qquad
            A = \begin{bmatrix}
                2 & 1 \\ 1 & 2
            \end{bmatrix}
        \]
        Determine uma aproximação para \(\frac{z_1}{z_2}\),
        tal que \(z = A^{1 \, 000 \, 000} \; x\).
    \end{question}
\end{exe}

\end{document}
